\documentclass[12pt]{article}
\usepackage{fullpage,enumitem,amsmath,amssymb,graphicx}

% Pour avoir les accents
\usepackage[utf8]{inputenc}  
\usepackage[T1]{fontenc}   
\usepackage{bbm}

% Some macros for your convenience
\newcommand\bbR{\ensuremath{\mathbb{R}}} % Real numbers
\newcommand\bbZ{\ensuremath{\mathbb{Z}}} % Integers
\newcommand\bbE{\ensuremath{\mathbb{E}}} % Expectation
\DeclareMathOperator*{\tr}{tr} % Trace
\DeclareMathOperator*{\diag}{diag} % Diagonal matrix
\DeclareMathOperator*{\sign}{sign} % Sign
\DeclareMathOperator*{\var}{Var} % Variance
\DeclareMathOperator*{\cov}{Cov} % Covariance
\DeclareMathOperator*{\argmin}{Argmin} % \argmin
\DeclareMathOperator*{\argmax}{Argmax} % \argmax

\newcommand{\1}{\mathbb{I}} % Indicator
%\newcommand{\newproblem}[1]{\newpage \section*{Problem #1}}

\begin{document}

\begin{center}
{\Large MAP556 PC1 Exercice 1}

\begin{tabular}{c}
Peng-Wei Chen
\end{tabular}
\end{center}
\section*{Question 1}
On identifie la densité de X comme $\lambda e^{-\lambda x} \mathbbm{1}_{x\ge 0}$. On montre qu'on peut appliquer la méthode de vraisemblance à cette question :
\begin{enumerate}
    \item La densité de $X_\lambda$ est strictement positive sur $[0, +\infty[$ par rapport à la mesure $\mu (dx) = dx$.
            \item La fonction $(\lambda, x)\in ]0, +\infty[ \times \mathbb{R} \mapsto p(\lambda, x) = \lambda e^{-\lambda x} \mathbbm{1}_{x\ge 0}$ est continûment différentiable par rapport à $\lambda$ avec $|\partial_{\lambda} p(\lambda, x)| = |(1 - \lambda x)e^{-\lambda x}|$. On peut trouver une majoration intégrable de $|\partial_{\lambda} p(\lambda, x)|$ sur un voisinage de $\lambda$ parce que c'est une fonction affine en x multipliée par $e^{-\lambda x}$.
    \item Par la définition de F, F est measurable et bornée.
\end{enumerate}
Ainsi, on a
\begin{equation*}
    \begin{array}[h]{r l}
f'(\lambda) & = \partial_\lambda \mathbb{E}\left[F(X)\right] = \mathbb{E}\left[ F(X)\partial_\lambda\left[log(p(\lambda, x)\right]|_{x = X}\right]\\
    &   = \int_0^{+\infty} F(x)\left( \frac{1}{\lambda} - x\right) \lambda e^{-\lambda x} dx\\
        
    \end{array}
\end{equation*}
\section*{Question 2}
On peut écrire $X = -\frac{1}{\lambda} log(U)$ où U est une loi uniforme sur $[0, 1]$. Alors, 
\[
    \partial_\lambda X = \frac{1}{\lambda^2}log(U) = -\frac{1}{\lambda}X
\]
On montre qu'on peut appliquer la méthode de dérivation à cette question :
\begin{enumerate}
    \item La fonction $\lambda \mapsto -\frac{1}{\lambda} log(U)$ est $C^1$ p.s. Et $\partial_\lambda X < -\frac{4}{\lambda^2} log(U)$ sur le voisinage $]\frac{\lambda}{2}, 2\lambda[$ de $\lambda$. La majoration est intégrable sur ce voisinage aussi.
    \item F est une fonction $C^{1}$ à dérivée bornée. (La question ne donne que $D^{1}$, mais cela n'est pas suffisant)
\end{enumerate}
Ainsi, on a
\begin{equation*}
    \begin{array}[h]{r l}
    f'(\lambda) & = \partial_\lambda \mathbb{E}\left[F(X)\right] = \mathbb{E}\left( \triangledown F(X)\partial_\lambda X \right)\\
    &   =\mathbb{E}\left( -F'(X)\frac{X}{\lambda} \right)\\
    &   = \int_0^{+\infty} -F'(x)\frac{x}{\lambda}\lambda e^{-\lambda x} dx\\
    &   = \int_0^{+\infty} -F'(x)xe^{-\lambda x} dx\\
    \end{array}
\end{equation*}

\section*{Question 3}
On va utiliser l'intégration par parties sur le résultat de la question 1. On prend $f(x) = F(x)$, $g_\lambda'(x) = \left( \frac{1}{\lambda} - x \right) \lambda e^{-\lambda x} dx$. Ainsi, on a 
\[
    g_\lambda (x) = xe^{-\lambda x}
\]
et donc
\begin{equation*}
    \begin{array}[h]{l}
    \int_0^{+\infty} F(x)\left( \frac{1}{\lambda} - x\right) \lambda e^{-\lambda x} dx\\
    = \left[ F(x)xe^{-\lambda x} \right]^{+\infty}_{0} - \int_0^{+\infty} F'(x) xe^{-\lambda x} dx\\
    = -\int_0^{+\infty} F'(x)xe^{-\lambda x} dx
        
    \end{array}
    \label{non}
\end{equation*}
où on retrouve le résultat de la question 2.

\end{document}
